\documentclass{article}
\usepackage{graphicx} % Required for inserting images
\usepackage{hyperref}
\newcommand\tab[1][0.5cm]{\hspace*{#1}}
\usepackage[letterpaper,top=2cm,bottom=2cm,left=3cm,right=3cm,marginparwidth=1.75cm]{geometry}

\title{Notes}
\author{Hugo Strappazzon}
\date{February 2024}

\begin{document}

\maketitle

\section*{Articles}
\subsection*{Displacement Interpolation Using Lagrangian Mass Transport}
\href{https://www.researchgate.net/publication/220183864_Displacement_Interpolation_Using_Lagrangian_Mass_Transport}{Link pdf}\\
\textbf{Mass Transport Problem} :\\
Transportation Simplex $->$ Earth Moving Distance\\
Network simplex algorithm (\textit{with block search pivoting ??}) from LEMON graph library $->$ general min-cost flow problems
Transportation simplex have worst case complexity in $O(n^3)$ but generally behaves in $O(n^2)$ in some context.\\

\noindent\textbf{Limitation} : Problem size $->$ cost matrix storage too heavy for GPU memory.
Presented method working for interpolation between two distributions, future work idea : interpolation between N distribution (texture mixing)

Radial basis function ?

\subsection*{Minimum-cost flow algorithms: An experimental evaluation}
\href{https://egres.elte.hu/tr/egres-13-04_September_2013.pdf}{Link pdf}\\
\textbf{Minimum-cost flow algorithms in the LEMON library :}\\

\noindent\textbf{Spanning tree data structures} :\\
\noindent ATI (Augmented Threaded Index ) vs  XTI (eXtended Threaded Index ) XTI apparently have better performance for network simplex.\\
Additional improvement : a reverse thread index is also stored for each node to represent the depth-first traversal as a doubly-linked list.\\
For initialization of the initial spanning tree solution, adding an artificial root with additionnals arcs between the nodes and the new roots provides better performances.\\

\noindent\textbf{Pivot rules :}\\
\noindent simplest pivot rules : \textit{best eligible} and \textit{first eligible}.\\

\noindent\textit{block search pivot rule} : cyclically examines certain subsets (blocks) of the arcs and select best candidate at each iteration. Block size seems quite important parameter, set the size proportionally to $|A(G)|$ between 1\% and 10\%. Article experiments suggest $block\_size = \lfloor\sqrt{m}\rfloor$ with (m = $|A(G)|$, the number of arc in graph $G$)\\

\noindent\textit{candidate list pivot rule} : method that examines arcs and build a list of most eligible arcs. The list is then used for at most $K$ iterations. If an arc becomes non-eligible, it is removed from the list. Article suggest : $L = \lfloor\sqrt{m}/4\rfloor$ and $K = \lfloor L/10\rfloor$.\\

\noindent\textit{altering candidate list pivot rule} : improved version of \textit{candidate list pivot rule} presented in the article. It maintain a list of size $K$ of eligible arcs at each iteration, extending and removing arcs only searching in one arc block with size $B$. Article suggest $B = \lfloor\sqrt{m}\rfloor$ and $K = \lfloor B/100\rfloor$.

\subsection*{A networksimplex algorithm with O( n) consecutive degenerate pivots}
\href{https://www.sciencedirect.com/science/article/abs/pii/S0377221716309493?casa_token=SMqEV_sxidsAAAAA:3CR4u27p_VI9Bk48yrlLN0HKp-sdVPk_YAORscbmIe7yMcQ-zb935FCKALHu8hFgY9NAUE_JDXY}{Link}

\textbf{if} there is an eligible arc (k; l) such that nodes k and l are in the same free component, then pivot in the arc (k; l);

\subsection*{A network simplex method for the budget-constrained minimum cost flow problem}
\href{https://www.sciencedirect.com/science/article/abs/pii/S0167637702001141?casa_token=Iofw98BgAS0AAAAA:1ULyjeo9cNWUvxOTDPwAj6DMAjnyZmUGLdUGYDWC3CJ_bJVmBRHRPKLiZ8yYPMHjEYUtgQfb5B8}{Link}


\section*{Basic simplex network for min-cost flow problem}
\textbf{Input : } Graph, cost matrix, capacity matrix, source(s) and sink(s)\\
Compute Initial feasible solution as spanning tree structure (T, L, U)\\
\textbf{do}\\
\tab Apply pivot rule $->$ give entering arc\\
\tab Apply leaving arc rule $->$ give leaving arc\\
\tab compute new flow\\
\tab update potential node and reduced cost\\
\textbf{until} optimality condition\\
\textbf{Output : } Optimal min-cost flow
\vspace{15px}\\
TODO : \\
change function to manage generic number\\
refactor \textit{compute\_node\_potential}\\
modify \textit{Push relabel}\\
modify algo to reduced max-flow to demand\\
implement other \textit{pivot rule}\\
Parallelism model\\



\newpage

\section*{Ressources}

\noindent repository of Optimized version of Simplex Network from LEMON : \url{https://github.com/nbonneel/network_simplex}\\

\noindent Simplex algo for mincost flow problem : \textit{Network Optimization: Continuous and Discrete Models, p. 201}\\

\noindent NETWORK FLOWS Theory, Algorithms, and Applications (p.417) \url{http://www.dl.behinehyab.com/Ebooks/NETWORK/NET005_354338_www.behinehyab.com.pdf}




\end{document}
